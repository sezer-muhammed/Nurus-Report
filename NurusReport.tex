\documentclass{article}
\usepackage[utf8]{inputenc}

\title{Sponsorship Report}
\author{Muhammed Sezer / Şevval Belkıs Dikkaya}
\date{\today}

\begin{document}

\maketitle

\section{Introduction}

In this report, we will detail our sponsorship agreement with NURUS, including the terms of the agreement, the benefits we received, and the results of our collaboration.

\section{Background}

Our university team was tasked with designing and building a fixed-wing aircraft for Teknofest. In order to create a high-quality UAV, we needed access to specialized equipment and materials. We reached out to NURUS, an innovative and R\&D-oriented company that produces office furniture, as well as works in the fields of aviation and technology, to explore the possibility of sponsorship.

After contacting Renan Gökyay, a representative of NURUS, we were able to secure a sponsorship package that included a budget of 20.000TL and access to their workspace and equipment. We used a variety of top-quality materials, including fabric, carbon, epoxy, and other composites, to build our fixed-wing aircraft.

\section{Sponsorship Agreement}

The sponsorship agreement began in December 2022 and will end in April 2023. As part of the agreement, NURUS required us to place their logos on the aircraft and team t-shirt, and to create a video in their factory.

\section{Benefits}

Through our collaboration with NURUS, we were able to benefit from their expertise and resources, including their workspace and equipment. We also received financial support.
Thanks to NURUS' sponsorship, we had access to a wealth of opportunities within their factory, which allowed us to create a superior aircraft compared to our competitors.

\section{Results}

As a result of our partnership with NURUS, we were able to create an exceptional aircraft that outperformed our competitors. We were also able to fulfill the requirements of the sponsorship agreement by placing NURUS' logos on our aircraft and t-shirt, and by creating a video in their factory.

\section{Factory Experience}

During our time at NURUS, we had access to their workspace and equipment, which allowed us to work on our aircraft in a professional setting. We were also able to collaborate with their team members who provided us with valuable insights and guidance on how to improve our aircraft.

Here are some of the steps we took during the process:

\begin{itemize}
    \item We used NURUS' specialized software to design and test our aircraft.
    \item We used their top-quality materials, including fabric, carbon, epoxy, and other composites, to build our aircraft.
    \item We used their tools and equipment to manufacture and assemble the aircraft, including a CNC machine, a laser cutter, and a 3D printer.
    \item We worked closely with their team members to troubleshoot any issues and optimize our design.  
	\item Access to NURUS' network of industry professionals and potential collaborators
	\item Exposure to new technologies and manufacturing processes, including additive manufacturing and composites fabrication
	\item Training sessions and workshops on topics such as design software, machining, and testing techniques
	\item Feedback and guidance from NURUS' experienced engineers and designers
	\item Opportunities to showcase your work and promote your team through NURUS' social media and marketing channels
	\item Invitations to attend industry events and conferences as representatives of NURUS and your university
	\item Potential for ongoing collaboration with NURUS on future projects or research initiatives
\end{itemize}

\section{Our Experience with NURUS}

Working with NURUS was an unforgettable experience that challenged us to push the limits of our knowledge and skills. From the moment we stepped into their workspace, we were surrounded by a team of passionate engineers and designers who were eager to help us succeed.

Our journey began with the design phase, where we used NURUS' specialized software to create a 3D model of our aircraft. With the help of their team members, we were able to refine our design and optimize it for performance and efficiency.

Once we had our design locked in, we began the manufacturing process. This was where NURUS truly shone, as we were given access to their state-of-the-art equipment and materials. We used their CNC machine to precisely cut our components, their laser cutter to create intricate patterns in our wings, and their 3D printer to produce small parts with extreme accuracy. We also used their high-quality materials, including carbon fiber, epoxy, and composites, to construct our aircraft.

As we worked on our aircraft, we were constantly supported by NURUS' team members, who provided us with invaluable feedback and guidance. We were encouraged to ask questions and seek help whenever we needed it, and we always felt like we were part of the NURUS family.

One of the most memorable moments of our experience was when we were invited to present our progress to NURUS' management team. We gave a comprehensive presentation on our design, manufacturing, and testing processes, and were met with enthusiastic applause and encouragement.

Throughout our time at NURUS, we also had the opportunity to learn about the company's other projects and initiatives, which ranged from cutting-edge furniture design to advanced aviation research. We were inspired by their commitment to innovation and excellence, and felt privileged to be part of their community.

As our project came to a close, we were proud to see our aircraft take shape and perform at its best. We knew that without NURUS' support, we would not have been able to achieve such a high level of success. We are grateful to NURUS for giving us this opportunity, and we look forward to maintaining our relationship with them in the future.


\section{Conclusion}

Our sponsorship agreement with NURUS was a success, and we are grateful for the support and resources they provided us. We look forward to the possibility of future collaborations with NURUS.

\end{document}
